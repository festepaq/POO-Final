Ésta es la idea que fui bosquejando de clases:

Juego:
Esta clase tiene información importante del juego, como una referencia al tablero, un stack de registro de jugadas, a quién le toca jugar, etc. También tiene métodos, los cuales serían usados externamente (por la interfaz gráfica, por ejemplo) para mandarle mensajes como qué casillero se clickeó o un pedido de revertir la última jugada. La clase sería algo así:

public class Juego {
	
	private Tablero elTablero; // Ésta es una referencia al tablero

	private Stack<Jugada> registro; // Ésta es una pila donde se guardan las jugadas a medida que se hacen

	bool jaqueMate;	// Una variable que registra si hubo jaque mate o no
	bool ahogado;	// Una variable que registra si hubo ahogado o no

	public String clickTablero(pos_x, pos_y);	// Con este método se le dice al juego qué posición del tablero fue clickeada.
												// Devuelve un String a la interfaz con la jugada, si es que hubo (ej: "7. Cb3")
												// Si no hubo jugada, devuelve null.
												// (No decidí cómo implementar las posiciones, podrían ser int)

	public void revertir();	// Revierte la última jugada hecha, sacándola de la pila del registro y mandándosela al tablero para
							// que la deshaga
	
	/* Falta algo que diga de quién es el turno, pero todavía no se me ocurrió si implementar clases 'Jugador' o simplemente decir 'blancas' o 'negras' */

	/* Otros métodos y miembros que hagan falta */
}

Tablero:
Esta clase tiene toda la información del tablero, como un arreglo de 8x8 casilleros, una referencia al casillero seleccionado y métodos para interactuar con él. Sería algo así:

public class Tablero {
	
	private Casillero[8][8] losCasilleros;	// Una matriz de casilleros, uno por cada escaque

	private Casillero seleccionado;	// Una referencia al casillero seleccionado (lo cual se hace haciéndole click)

	public Jugada click(pos_x, pos_y);	// Este método lo usa el juego para decirle qué casillero fue clickeado
										// Entonces el tablero decide en base al casillero clickeado y al seleccionado
										// (anteriormente) si debe hacerse una jugada o no. Si se hace, se devuelve en el
										// método y se pone seleccionado en null; si no se hace, se pone seleccionado en el que // se clickeó ((pos_x, pos_y) sería) y se devuelve null.

	void revertir(Jugada laJugada);		// A ḿétodo lo llama 'revertir()' de la clase Juego. Le pasa la jugada para que la
										// deshaga (fíjense que el tablero no sabe cuál fue la última jugada, entonces hay que
										// decírselo)

	/* Otros métodos y miembros que hagan falta */
}

Casillero:
Esta clase representa un escaque del tablero. Tiene que saber si está ocupado por una pieza, y si lo está, por cuál. Se tiene que poder darle una pieza o vaciarlo (para mover las piezas por el tablero). Todavía no decidí si vale la pena que el casillero sepa en qué parte del tablero está, ya que esta información ya la tiene el tablero al indexarlo en su matriz; sin embargo, 'seleccionado' de la clase Tablero tendría que saber dónde está y está por fuera de la matriz. Algo así pensé:

public class Casillero {
	
	private Pieza laPieza;	// Una referencia a la pieza que tiene encima, que es null si no hay ninguna

	// private pos_x, pos_y ?

	public bool estaOcupado();	// Devuelve si hay o no una pieza

	public void darPieza(Pieza unaPieza);	// Sirve para ponerle una pieza al casillero

	public void darPieza(Casillero otroCasillero);	// Sirve para darle la pieza de otro casillero

	public void vaciar();	// Saca la pieza del casillero (pone laPieza en null escencialmente)

}

Pieza:
Esta clase es la subclase para todas las piezas. Tiene información básica de la pieza que todas tienen. Es abstracta porque no hay nada que sea sólo una pieza: Esto es lo que pensé:

public abstract class Pieza {
	
	Private Color elColor;	// Algo que diga si es blanca o negra, aunque no sé si es la mejor forma de implementarlo

	public Color dameColor();	//	Trivial

	public abstract Movimiento dameMovimiento();	// La idea sería que cada pieza sepa cómo moverse y que se lo pueda decir
													// decir al tablero en caso de ser clickeada. No decidí si implementar
													// una clase Movimiento sería la mejor idea, tendremos que discutirlo.

	/* Otros métodos y miembros que hagan falta */

}

Peón, Caballo, Torre, Alfil, Dama, Rey:
Estas clases heredan de Pieza y son concretas. Tienen que implementar el método dameMovimiento() y saber cómo moverse para decírselo al tablero. El tablero decidirá en base a eso a dónde pueden ir esas piezas.

Jugada:
Esta clase debe tener toda la información de una jugada: qué se movió y a dónde. No pensé cómo implementarla.

De la interfaz gráfica no pensé nada, sólo que va a haber un tablero y que se juega haciendo click (no arrastrando las piezas con el mouse). Creo que lo mejor sería que nos dividiéramos lo mejor posible el trabajo, que alguien (que no sea yo, por favor)se ocupe de la parte gráfica en Swing (una o dos personas) y los demás trabajemos en la lógica del juego. Me parece que si todos trabajamos de forma prolija, deberíamos poder desarrollar el fron-end y el back-end por separado y después acoplarlo fácilmente. Eso es todo.
Todo ok.